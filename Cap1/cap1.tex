Over the past decade, advances in cloud computing, open source ecosystems, and software development tooling have significantly reduced the cost and complexity of launching digital products. These enablers have contributed to a surge in software startups that are capable of reaching global scale to onboard thousands or millions of new customers in days or months instead of years or decades like a traditional business usually takes during maturing stage; however, this acceleration also introduces new forms of complexity. In particular, startups operate under critical resource constraints and must validate their business models quickly, facing conditions that amplify the long-term impact of early technical decisions. Among these, the choice of software architecture plays a pivotal role in determining a product’s ability to evolve, scale, and remain maintainable in the long term. This research investigates a key architectural trade-off within the domain of cloud native systems: whether to adopt a modular monolithic architecture or a microservices-based design during the early stages of a software development in startups. To establish the foundation for this investigation, the following sections examine the defining characteristics of software startups, articulate the motivations underpinning the research, and present the formal research objectives and guiding questions.

\section{Contextualization}
Startup entrepreneurs often face significant challenges when transforming ideas into successful products; statistics reveal that only one in ten startups ultimately succeeds. Self-inflicted issues, rather than external competition, are the primary reasons why most startups fail within two years of creation \cite{crowne2002}. Approximately 20\% fail within the first year, and 70\% fail within two to five years. Contrary to common assumptions among first-time founders, lack of funding is rarely the main cause of failure. Furthermore, only 25\% of venture-backed startups deliver any return on capital, and in 30\% to 40\% of failure cases, investors lose their entire initial investment \cite{demandsage2023}.

To contextualize these challenges, it is important to understand what defines a startup and distinguishes it from other types of traditional businesses. Ries \cite{ries2011}, in \textit{The Lean Startup}, defines a startup as a human institution designed to create new products or services under conditions of extreme uncertainty. Similarly, Blank \cite{blank2012} describes a startup as a temporary organization searching for a repeatable and scalable business model, emphasizing growth as a defining characteristic.

Unlike small businesses that may not prioritize growth, or miniature versions of large business corporations, startups aim to scale rapidly once they reach product-market fit, a stepping stone at which they identify a target audience, understand customer needs, and deliver a product that effectively addresses those needs in the market \cite{ries2011}. This potential for rapid scaling sets startups apart from other traditional business models.

Technological advances such as enhanced internet infrastructure, cloud computing, artificial intelligence, API integrations, low-code platforms and modern web and mobile frameworks have dramatically lowered the barrier to launching startups. High-quality entrepreneurial education is now widely available online and often free. However, founders still face limited resources, tight deadlines and minimal funding while competing in highly dynamic markets \cite{paternoster2014}, creating new pressures. In this environment, software offers a uniquely scalable form of leverage: once written, code can be deployed repeatedly at near-zero marginal cost. Ravikant \cite{ravikant2020}, talks about \textit{“1,000x engineers”} or \textit{“leveraged workers”}, who generate immense value with a few lines of well-crafted code.\footnote{Naval Ravikant defines leverage as tools that multiply effort: code, media and capital. Code, in particular, allows a single engineer to reach millions of users with minimal marginal cost \cite{ravikant2020}} These founding engineers demonstrate that in software, impact and financial success are not necessarily proportional to effort and hours of coding.


Examples of such figures include Markus \textit{\enquote{Notch}} Persson (Minecraft), Satoshi Nakamoto (Bitcoin), and John Carmack (id Software), who transformed entire industries, creating billions of dollars in value through solo code contributions. Founders such as Jeff Bezos, Mark Zuckerberg, Larry Page, Sergey Brin, Bill Gates, and Steve Jobs exemplify how sustained code-based leverage can fuel exponential organizational growth across multiple cycles.

Yet, as Ravikant \cite{ravikant2020} cautions, even ten talented engineers can waste valuable effort if they pursue the wrong model or build the wrong product. Outcomes in software startups must rely on strategic clarity rather than brute force. This reflects Howard Marks' insight in \textit{The Most Important Thing} \cite{marks2013}, where he notes that probable outcomes fail to materialize all the time, while improbable successes occur regularly. Startups are usually improbable successes that come from nowhere.

In the case of software startups, initial traction often appears unlikely. Nevertheless, when a product aligns with genuine market demand and reaches users effectively, it can trigger exponential growth \cite{graham2012}, as talent, capital, and word-of-mouth acquisition converge to accelerate momentum, but this growth is delicate; poor decisions can stall or reverse it. To measure and sustain that momentum, startups must adopt a data-driven mindset: prioritize actionable metrics that reflect real customer behavior, use a North Star Metric to align the team, and leverage KPIs and OKRs to set and track goals around engagement, retention, and referrals, avoiding vanity metrics that offer little strategic insight \cite{blank2012, doerr2018measure, bernstein2020kpi}.

One common pitfall arises when founders shift focus away from core areas such as product development, engineering, sales, and customer support to pursue external validation, often over-relying on advice from investors or corporate advisors who bring large-organization thinking; as a result, startups may over-hire, over-engineer, introduce rigid structures, and dilute their agility, causing inefficiencies that lead to stagnation and erode the startup’s leverage. To avoid this, companies must stay lean and focused: as Airbnb’s founder Brian Chesky advises \cite{chesky2024}, they should remain operationally flat and expertise-driven so that founders stay close to key operations, guide product cycles, and sustain momentum, because losing this connection only further weakens execution.

To answer this type of dilemma, startups must make deliberate, evidence-based decisions in both business and technology. Weinberg \cite{weinberg2015} argue for a balanced approach where founders allocate equal attention to product engineering and distribution, such as marketing or sales initiatives. Great products alone do not guarantee success, nor does strong marketing without software engineering excellence. Both must progress in harmony to enable sustainable growth.

In summary, just as poor business choices can limit a startup's trajectory, misguided technical decisions such as adopting overly complex software architectures or chasing hyped technologies can introduce significant engineering debt. These choices undermine maintainability and delay new features' deployments, proving that strategic discipline is required in both product and technology decisions for a successful startup. Founders must maintain participation, focus on actionable metrics, and prioritize long-term agility in order to survive. The same discipline must apply to technical choices.

\section{Motivation}

While software engineering trends evolve quickly, blindly adopting fashionable architectures or tools can backfire. Complex, over-engineered systems introduce technical debt and slow down development cycles. Therefore, architectural decisions must align with long-term goals rather than short-term trends. Startups benefit from frameworks that emphasize simplicity, maintainability, and flexibility, particularly when resources are limited and product-market fit is still emerging. This perspective aligns with industry critiques about the architectural choices made by immature engineering teams. As David Heinemeier Hansson (DHH) states, \textit{\enquote{If you can't build a well-structured monolith, what makes you think microservices are the answer?}} \cite{dhh2021}

Software startups face critical architectural choices early in their development. Some adopt microservices from the outset, seeking modularity and scalability but incurring high complexity in deployment, orchestration, and team coordination. Others begin with a monolithic approach to prioritize speed and simplicity, yet often encounter significant obstacles when evolving their systems to meet growing demands.

A recurring challenge in both paths is achieving and sustaining \textbf{modularity}. While modularity is essential for maintainability, scalability, and team autonomy, it is rarely addressed in a structured or systematic way. In monolithic systems, in particular, the absence of internal modular boundaries can lead to accumulating technical debt and hinder architectural evolution. This raises a critical question for software teams: \textbf{to avoid premature adoption of microservices, as monolithic applications evolve, what architectural aspects must be evaluated to ensure modularity is preserved and the system remains distribution-ready?}

Despite growing recognition of these architectural tensions and new challenges in the cloud computing era to develop reliable and scalable applications, there are still no widely adopted \textit{ guidelines} for how to progressively evolve the architecture of software, especially monolithic applications. Startups and engineering teams lack concrete criteria to assess when and how to introduce modular structures, split components, or transition to distributed systems, including microservices.

These challenges underscore the need for a software architectural approach that aligns with the realities of startups, one that is flexible, scalable, and grounded in real-world constraints. This research seeks to explore how modular monoliths can meet that need and what is required to write modular and scalable cloud-native applications through this architecture.


\section{Objective}

The primary objective of this research is to \textbf{propose a set of architectural guidelines that position modular monolithic architectures as a pragmatic starting point to build scalable and maintainable cloud-native software applications in startups}. Rather than treating modular monoliths as an end state of software design and architecture implementation, the research results shall be a set of guidelines that emphasize their role in supporting early-stage development while preserving the structural readiness to evolve into microservices-based architectures as complexity and demand increase. By synthesizing architectural principles, evaluation criteria, and actionable guidelines, the study seeks to help software startups avoid premature architectural complexity, reduce technical debt, and maintain agility in resource-constrained environments.

To achieve this goal, the study pursues the following specific objectives:

\begin{enumerate}
\item \textbf{Analyze the causes of architectural failures in startups:} Investigate why many startups adopt over-engineered distributed systems or create unmaintainable monoliths, despite initial intentions to build scalable and modular software.

\item \textbf{Identify implementation challenges for modular monoliths:} Examine technical, organizational, and cultural obstacles that hinder successful adoption of modular monolithic architectures.

\item \textbf{Establish architectural principles and evaluation criteria:} Define principles and criteria that support maintainability, scalability, and readiness for service distribution, drawing from academic literature and industry practices.

\item \textbf{Design structured architectural guidelines:} Develop a coherent set of prescriptive guidelines that support the construction and evolution of modular monoliths within startup software contexts.

\item \textbf{Validate the guidelines with expert feedback:} Assess the clarity and applicability of the proposed architectural guidelines through structured engagement with experienced practitioners, such as interviews or expert surveys, rather than production-level implementation.
\end{enumerate}

\section{Methodology}

This research proposal adopts a multi-phase methodological approach that combines evidence-based synthesis, artifact design, and expert validation to investigate modular monolithic architecture as a pragmatic alternative to microservices in cloud-native software development. The research is structured into three sequential phases, each designed to contribute to the formulation and refinement of a set of architectural guidelines aimed at helping software startups design scalable, maintainable systems that retain internal modularity and can evolve into distributed architectures when justified by real-world growth and demand.

\textbf{Phase 1 – Systematic Literature Review (SLR):}  
The first and current phase of this research consists of a Systematic Literature Review (SLR), which serves as the empirical foundation of the study. The objective is to synthesize the state of academic knowledge on modular monolithic architectures, their benefits and limitations in comparison to microservices, and the architectural trade-offs they entail, particularly in relation to scalability, maintainability, and early-stage development.

The review was conducted in accordance with the guidelines for evidence-based software engineering proposed by Kitchenham and Charters~\cite{kitchenham2007guidelines}, ensuring transparency, reproducibility, and methodological rigor. The process involved the formulation of research questions, a structured search strategy, definition of inclusion and exclusion criteria, and a multi-step screening and synthesis protocol. The scope was restricted to peer-reviewed journal articles and conference papers published between 2019 and 2024, retrieved from two primary databases: IEEE Xplore and the ACM Digital Library. The resulting set of 18 studies was evaluated using a structured set of architectural criteria to identify recurring trade-offs and knowledge gaps. The insights from this review support the development of the architectural guidelines proposal available in Chapter~\ref{sec:proposal}, with a detailed synthesis of the initial findings available in Chapter~\ref{sec:relatedwork}.

\textbf{Phase 2 – Guidelines Development (to be done):}  
The second phase of this research will involve the formulation of a set of architectural guidelines specifically tailored for software startups adopting modular monolithic architectures. These guidelines will be structured to address the twelve evaluation criteria identified in the literature review, providing targeted recommendations for design decisions that affect modularity, maintainability, scalability, and migration readiness. The output of this phase will be a prescriptive, criteria-based decision-making set of guidelines that helps teams assess trade-offs and implement modular monoliths in a way that supports future distribution into microservices when needed. The artifact will be grounded in the principles of scientific research and will synthesize the SLR findings into a clear, actionable format suitable for use in early-stage, cloud-native development environments.


\textbf{Phase 3 – Expert Validation (to be developed):}  
The third and final phase of this research, to be carried out in the next stage of the research, will focus on validating the proposed guidelines through engagement with experienced software practitioners. This validation will involve structured interviews or review sessions designed to assess the clarity, feasibility, and practical relevance of the guidelines in real-world development environments.

Participants will include professionals with expertise in software architecture, particularly those working in startups or cloud-native contexts. In addition, the study will seek to incorporate insights from practitioners at companies known for adopting modular monolithic architectures such as Shopify, GitLab, and GitHub, as their experience can provide valuable lessons on the practical trade-offs and long-term implications of this approach. Feedback gathered during this phase will guide refinements to the guidelines, ensuring that they reflect both academic principles and industry realities before final consolidation.

\medskip

By structuring the research across these three phases, the study aims to ensure both academic rigor and practical relevance. The first phase establishes a solid empirical foundation; the second phase, to be developed, will produce an artifact containing clear and specific architectural guidelines; and the third phase will provide external validation through expert review. This multi-method approach strengthens the credibility, clarity, and utility of the final academic and industry contribution the author aims to produce.

\subsection*{Significance of the Study}

This research provides a clear and actionable set of architectural guidelines for leveraging modular monolithic architectures as a pragmatic foundation in software startups. It aims to assist engineers in avoiding the premature complexity of microservices and the long-term maintenance issues of poorly structured monoliths. The proposed guidelines enable the development of adaptive, progressively scalable and maintainable systems that facilitate iterative growth, control technical debt, and reduce the likelihood of costly rewrites as product demands evolve. Beyond its immediate relevance to startups, this study contributes to the broader discourse on architectural decision-making in cloud-native environments, offering principles and guidance applicable across a range of organizational contexts.

\pagebreak
\section{Document Structure}

This document is structured to progressively build the foundation for architectural guidelines that support the design and evolution of modular monolithic architectures in software startups. The current chapter introduces the research motivation, defines the primary and specific objectives, presents the methodological approach, and discusses the significance of the study within the context of cloud-native software development and early-stage product environments.

Chapter~\ref{sec:background} provides the theoretical background and conceptual foundations necessary to understand the architectural paradigms explored in this work. It introduces and defines key concepts such as cloud-native systems, microservices, and modular monoliths, while examining how technical and organizational dynamics influence architectural decisions in startup contexts.

Chapter~\ref{sec:relatedwork} presents the full execution of the systematic literature review. It details the search strategy, inclusion and exclusion criteria, bibliographic databases used, and the multi-stage screening process applied to select relevant peer-reviewed studies. The chapter concludes with a synthesis of findings from the selected literature, highlighting architectural trade-offs, recurring challenges, and implementation strategies related to modular monolithic systems. These insights form the empirical foundation for the research proposal.

Chapter~\ref{sec:proposal} presents the research proposal, including the formal problem definition and the planned development of a set of architectural guidelines to guide the design of modular monolithic applications. The guidelines aim to support maintainability, scalability, and readiness for future distribution into microservices when justified by actual demand, providing a pragmatic alternative to premature adoption of microservices architecture (MSA).

Finally, Chapter~\ref{sec:conclusion} concludes the research proposal by summarizing the potential contributions of the future research, it's limitations, and presenting the next steps in the study. This includes the planned design of the guidelines and the methodology for expert validation, setting the stage for the final phase of the research.
