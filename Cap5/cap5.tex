\label{sec:conclusion}

This chapter completes the research proposal by outlining the forthcoming stages of investigation, the expected contributions of the study, and the boundaries that will shape its development. The previous chapters established a structured foundation for examining modular monoliths as a viable architectural choice for early-stage cloud-native systems. From the systematic literature review, a set of twelve preliminary guidelines was synthesized: not as definitive answers, but as hypotheses grounded in academic literature and the recurring dilemmas observed in practice.

The intention behind Chapter~\ref{sec:proposal} was not to finalize these guidelines, but to present them as structured research propositions. The next phase of this work will focus on developing, refining, and empirically testing each of them through a combination of expert validation, continued theoretical expansion, and iterative modeling. This chapter articulates how that progression will unfold.

\section{Extending the Research Process}

The next step is to conduct the verification plan outlined previously. Semi-structured interviews with industry practitioners and academic researchers will confirm, refine or extend the guidelines and reveal any additional constraints or trade-offs. Insights from these interviews, combined with targeted research in the gray literature, will inform revisions, leading to a more actionable and context-informed set of heuristics to this research.

Through this reflective cycle, the guidelines will transition from literature-derived hypotheses to field-validated tools, ensuring clarity, applicability, and relevance in real-world startup environments. 

\pagebreak
\section{Scope and Limitations}

The research deliberately focuses on early-stage, cloud-native software startups, contexts where architectural decisions are often made under uncertainty, with limited time, infrastructure, and people. This scope excludes large-scale enterprise systems, highly regulated industries, or long-established software organizations, where architectural dynamics may differ substantially.

The guideline set itself is not intended to function as a prescriptive framework. Rather, it is structured as a series of modular principles, subject to validation, contextualization, and adaptation. The limitations of this work include the availability and diversity of expert participants, the challenge of generalizing between varying startup cultures and team configurations, and the interpretive nature of qualitative analysis. These risks will be mitigated through transparent methodology, critical triangulation, and clear scope delimitation.

\section{Final Remarks}

This proposal positions architectural guidance not as a static artifact, but as an evolving construct, shaped by both empirical evidence and contextual complexity. The guidelines introduced here reflect the first iteration of that construct. The next phase of this research will move beyond synthesis into engagement, testing whether these principles can meaningfully support the design and evolution of systems that remain simple in structure, scalable in function, and sustainable in team practice.

