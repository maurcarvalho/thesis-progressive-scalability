Startups de software cloud-native enfrentam uma tensão arquitetural fundamental: microsserviços introduzem complexidade operacional prematura que sobrecarrega equipes pequenas, enquanto monolitos tradicionais degradam-se em sistemas fortemente acoplados que progressivamente resistem a mudanças. Esta pesquisa aborda a lacuna entre esses extremos ao desenvolver seis diretrizes arquiteturais acionáveis (G1--G6) para escalabilidade progressiva em aplicações monolíticas modulares.

Uma revisão sistemática da literatura de 18 estudos revisados por pares, publicados entre 2019 e 2024, identificou 12 critérios de avaliação organizados em quatro dimensões (Design Arquitetural, Adequação Operacional, Alinhamento Organizacional e Orientação de Diretrizes) e revelou que nenhum estudo existente aborda diretamente a escalabilidade progressiva como uma abordagem de diretrizes passo a passo para monolitos modulares. Embora a literatura revisada cubra técnicas individuais de forma isolada, nenhum as integra em um conjunto progressivo e coeso de diretrizes. A contribuição central desta pesquisa é precisamente essa integração: a progressão G1$\rightarrow$G6 sintetiza técnicas estudadas individualmente em um caminho deliberado de escalabilidade.

As seis diretrizes são: G1 (impor limites modulares por meio de verificação automatizada), G2 (incorporar manutenibilidade como custo limitado de mudança), G3 (projetar para escalabilidade progressiva em um espectro de quatro níveis L0--L3), G4 (promover prontidão para migração utilizando Apache Kafka e Temporal), G5 (otimizar estratégia de implantação por meio de um espectro D0--D2) e G6 (introduzir padrões de observabilidade com OpenTelemetry). Juntas, elas garantem que a extração de módulos se torne uma mudança de topologia de implantação, e não uma reescrita arquitetural.

As diretrizes são fundamentadas no Tiny Store, uma implementação de referência funcional construída como um monorepo Nx com quatro contextos delimitados, 127 testes aprovados e uma infraestrutura de produção abrangendo 13 serviços conteinerizados. A verificação por meio de entrevistas semiestruturadas com profissionais especialistas está planejada como a próxima fase da pesquisa.

\vspace{1.0cm}
\noindent\textbf{Palavras-chave:} Arquitetura de Software. Monolito Modular. Escalabilidade Progressiva. Cloud-Native. Microserviços. Revisão Sistemática da Literatura.
