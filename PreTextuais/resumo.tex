Startups de software operam sob fortes restrições, com recursos financeiros limitados, alta incerteza e pressão constante por validação rápida no mercado. Nesse contexto volátil, decisões técnicas tomadas nas fases iniciais, especialmente no que se refere à arquitetura de software, têm implicações de longo prazo sobre a escalabilidade e manutenção dos sistemas. Neste cenário, arquiteturas baseadas em microserviços são amplamente promovidas como padrão para alcançar modularidade e escalabilidade, no entanto, introduzem sobrecarga operacional significativa e exigem um nível de maturidade organizacional que muitas equipes em estágio inicial ainda não possuem. Já monolitos tradicionais oferecem simplicidade inicial, mas frequentemente evoluem para sistemas fortemente acoplados, que tornam mais lentas as entregas e aumentam a complexidade e a propensão a falhas.

Esta pesquisa propõe arquiteturas monolíticas modulares como uma alternativa pragmática e ainda pouco explorada. Quando projetados com modularidade clara e disciplinada desde o início, esses sistemas oferecem a separação de responsabilidades necessária à evolução sustentável, ao mesmo tempo em que evitam fragmentação, complexidade de implantação e altos custos de coordenação associados à adoção precoce de microserviços.

A literatura existente aborda com profundidade tópicos como escalabilidade, manutenibilidade e gestão da complexidade, mas geralmente assume o uso de microserviços como premissa. Outras abordagens, especialmente os monolitos modulares, recebem atenção desproporcionalmente menor, o que gera lacunas conceituais e práticas relevantes, particularmente em tópicos como modularização, escalabilidade horizontal, onboarding de desenvolvedores e estratégias de implantação em ambientes de produção.

Como resposta, esta pesquisa apresenta um conjunto preliminar de diretrizes arquiteturais para apoiar a escalabilidade progressiva em aplicações monolíticas modulares. Essas diretrizes ajudam equipes a construir sistemas que preservem flexibilidade arquitetural ao longo do tempo, minimizando a complexidade desnecessária no curto prazo. A contribuição desta pesquisa é dupla: uma síntese conceitual que reposiciona os monolitos modulares como estratégia viável de longo prazo e um conjunto de diretrizes contextuais, baseadas em critérios arquiteturais, alinhadas à realidade do desenvolvimento de software sob restrições típicas de startups.