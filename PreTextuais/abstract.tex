Cloud-native software startups operate under significant constraints, including limited financial resources, high uncertainty, and pressure for rapid market validation. Within this volatile environment, early technical decisions, particularly those related to software architecture, carry long-term consequences for scalability, maintainability, and adaptability. While microservices architectures are widely promoted as the default model for achieving modularity and scalability, they introduce considerable operational overhead and require a level of organizational maturity that many early-stage teams have yet to develop. Conversely, traditional monoliths offer initial simplicity but often evolve into tightly coupled systems that progressively slow down feature releases and increase system complexity.

This research proposes modular monolithic architectures as a pragmatic and underexplored alternative for cloud-native systems in their formative stages. When designed with disciplined modularity, modular monoliths can provide the separation of concerns and internal boundaries necessary for sustainable software evolution. At the same time, they help teams avoid the fragmentation, deployment complexity, and coordination costs that frequently result from premature microservices adoption.

According to existing literature, key software topics such as scalability, maintainability, and complexity management are often discussed in depth when microservices is assumed as the architectural baseline. Other architectural approaches, particularly modular monoliths, receive disproportionately less attention, leading to conceptual and practical gaps. These gaps are especially evident in areas like modularization, horizontal scalability, maintainability, disposability, developer onboarding, architecture-team alignment, and deployment strategies that suit monolithic systems with strong modular structure.

In response, this research proposes a preliminary set of architectural guidelines aimed at supporting progressive scalability in modular monolith applications. These guidelines are intended to help teams structure monolithic systems in ways that preserve architectural flexibility over time while minimizing unnecessary complexity in the short term. The contribution lies in both a conceptual synthesis that repositions modular monoliths as a viable long-term strategy and a set of contextual, criteria-based guidelines grounded in the realities of software development under startup constraints.