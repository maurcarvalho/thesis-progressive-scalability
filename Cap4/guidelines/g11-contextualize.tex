\pagebreak
\section{G11: Contextualization}
\label{sec:g11-contextualize-to-constraints}

In this section, we present G11, which focuses on contextualizing architectural decisions to startup-specific constraints such as team size, funding, and operational maturity. G11 ensures that the chosen architecture fits the reality of the organization rather than imposing mismatched ideals.

\subsection*{Conceptual Overview}
To be done during research.

\subsection*{Objectives}
\begin{itemize}[noitemsep,topsep=2pt]
  \item \emph{Assess team size:} To be done.
  \item \emph{Evaluate funding constraints:} To be done.
  \item \emph{Match operational maturity:} To be done.
\end{itemize}

\subsection*{Key Principles}
\begin{itemize}[noitemsep,topsep=2pt]
  \item \emph{Principle 1:} To be done.
  \item \emph{Principle 2:} To be done.
  \item \emph{Principle 3:} To be done.
  \item \emph{Principle 4:} To be done.
  \item \emph{Principle 5:} To be done.
\end{itemize}

\subsection*{Metrics \& Verification}
\begin{itemize}[noitemsep,topsep=2pt]
  \item \emph{Constraint alignment score:} To be done.
  \item \emph{Resource utilization ratio:} To be done.
  \item \emph{Maturity correlation metric:} To be done.
  \item \emph{Decision outcome traceability:} To be done.
\end{itemize}

\subsection*{Documentation Guidelines}
\begin{itemize}[noitemsep,topsep=2pt]
  \item \emph{ADR Template:} To be done.
  \item \emph{Constraint assessment record:} To be done.
  \item \emph{Change Log Practices:} To be done.
\end{itemize}

\subsection*{Tool Capabilities}
Any open‐source or proprietary software tool used to support G11 should address:
\begin{itemize}[noitemsep,topsep=2pt]
  \item \emph{Constraint analysis dashboard:} To be done.
  \item \emph{Resource tracking integration:} To be done.
  \item \emph{Maturity evaluation tool:} To be done.
  \item \emph{Recommendation engine:} To be done.
  \item \emph{Scenario simulation:} To be done.
\end{itemize}

\subsection*{Literature Support Commentary}
Academic treatments of architecture rarely consider the unique constraints of early-stage startups; most assume large teams and mature processes. Practitioner retrospectives highlight these gaps, noting that decisions must account for limited headcount, tight budgets, and evolving requirements~\cite{su2024from}. G11 addresses this by embedding constraint analysis such as team size metrics, funding cadence, and maturity assessments into architectural trade-offs, grounding design in feasible startups realities.
