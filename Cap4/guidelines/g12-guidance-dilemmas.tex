\pagebreak
\section{G12: Structure Guidance Around Dilemmas}
\label{sec:g12-structure-guidance-around-dilemmas}

In this section, we present G12, which advocates structuring architectural guidance around dilemmas rather than prescriptive doctrines. By framing design choices as context-sensitive trade-offs, G12 allows teams to adapt principles to their particular environment, making guidelines both flexible and decision-driven.

\subsection*{Conceptual Overview}
To be done during research.

\subsection*{Objectives}
\begin{itemize}[itemsep=8pt,topsep=2pt]
  \item \emph{Identify common dilemmas:} To be done.
  \item \emph{Formulate adaptable principles:} To be done.
  \item \emph{Provide decision criteria:} To be done.
\end{itemize}

\subsection*{Key Principles}
\begin{itemize}[itemsep=8pt,topsep=2pt]
  \item \emph{Principle 1:} To be done.
  \item \emph{Principle 2:} To be done.
  \item \emph{Principle 3:} To be done.
  \item \emph{Principle 4:} To be done.
  \item \emph{Principle 5:} To be done.
\end{itemize}

\subsection*{Metrics \& Verification}
\begin{itemize}[itemsep=8pt,topsep=2pt]
  \item \emph{Dilemma resolution rate:} To be done.
  \item \emph{Principle adaptation score:} To be done.
  \item \emph{Decision outcome traceability:} To be done.
  \item \emph{Contextual applicability index:} To be done.
\end{itemize}

\subsection*{Documentation Guidelines}
\begin{itemize}[itemsep=8pt,topsep=2pt]
  \item \emph{ADR Template:} To be done.
  \item \emph{Dilemma catalog:} To be done.
  \item \emph{Change Log Practices:} To be done.
\end{itemize}

\subsection*{Tool Capabilities}
Any open‐source or proprietary software tool used to support G12 should address:
\begin{itemize}[itemsep=8pt,topsep=2pt]
  \item \emph{Decision-tracking tool:} To be done.
  \item \emph{Context simulation:} To be done.
  \item \emph{Guidance visualization:} To be done.
\end{itemize}

\subsection*{Literature Support Commentary}
Most architecture literature offers fixed rules such as \textit{ 'do X, avoid Y'} without acknowledging the situational environment. G12 remedies this by cataloging common startup dilemmas (e.g., “speed vs. consistency,” “centralization vs. distribution”) and providing decision criteria rather than one-size-fits-all mandates. This ensures guidance remains adaptable to real-world constraints and uncertainties.
